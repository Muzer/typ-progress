\documentclass[a4paper,12pt]{article}
\usepackage{hyperref}
\title{A real time train information and prediction system for the London Underground --- progress report}
\author{Murray Colpman --- Supervisor: Nick Gibbins}
\begin{document}
\begin{titlepage}
  \begin{center}
    \textsc{\Large Electronics and Computer Science}\\
    \textsc{\Large Faculty of Physical Sciences and Engineering}\\
    \textsc{\Large University of Southampton}\\[1.5cm]
    \textsc{\Large Murray Colpman}\\
    \textsc{\Large \today}\\[1.5cm]
    \textsc{\LARGE A real time train information and prediction system for the London Underground}\\[1.5cm]
    \textsc{\large Project supervisor: Nick Gibbins}\\
    \textsc{\large Second examiner: Iain McNally}\\[1.5cm]
    \textsc{\large A project progress report submitted for the award of}\\
    \textsc{\large MEng Computer Science --- 4443}
  \end{center}
\end{titlepage}

\section*{Abstract}

The London Underground currently lacks publicly-available software allowing its
users to track individual trains through the system. This project aims to
resolve this by creating a system to allow the user to view a list of trains
due to depart each station along with ones already departed, and to allow the
user to select a train to view its course throughout the system. So far,
background research has been performed, the overall design of the system has
been decided and a basic implementation of a tool to gather the data that will
be necessary to reconstruct the timetable has been produced, to run over the
coming months. Most of the implementation is still to be performed, including
producing an algorithm to reconstruct the timetable based on the real train
movements over a number of days, and a web interface to navigate around the
system. The stretch goals are to produce a system to predict times based on
actual performances of trains, and to produce a map of track codes (signalling
blocks that trains can occupy) based on the gathered data to further improve
predictions.

\pagebreak

\tableofcontents

\pagebreak

\section*{Statement of Originality}

This is all my own work except where explicitly indicated otherwise, and all
sources have been correctly acknowledged.

\pagebreak

\section{Project Description and Background}

\subsection{Realtime Trains --- an information system for the national railway network}

Rail enthusiasts make use of a website called Realtime Trains. For what is
generally referred to as the National Rail network, on which the vast majority
of heavy rail services in Great Britain run, this website makes use of open
data mainly from four sources, all provided by Network Rail (the public body
that owns and maintains the infrastructure and operates signalling
equipment for nearly all of the network)\cite{RTTData}:

\begin{itemize}
  \item TRUST, providing push data about trains passing certain timing points
    (mostly stations and junctions)
  \item TD, push data providing the train describer berth (which corresponds to
    a signalling section) that a given train is occupying --- cross-referenced
    with TRUST data to give more accurate predictions and platform alterations
  \item Schedule, pull data providing timetables for passenger and freight
    trains including changes to usual services (for engineering works, for
    example)
  \item VSTP, push data providing very short notice alterations to schedules
\end{itemize}

Realtime Trains uses these data sources to give views for the general public to
let people know where their train is, and to predict arrival times. It also has
a detailed view, allowing enthusiasts to see non-passenger trains, trains not
booked to stop at the station, detailed information about the routes trains are
booked to take (and actually take), and more detailed timing points. This can
be used by rail enthusiasts to track the position of a train they're interested
in, to track the train they're currently riding on to work out delays, or to
just see if any interesting trains are passing through a station.

\subsection{Adapting the concept to the London Underground}

The London Underground, despite having a few areas of overlapping operation, is
not generally a part of the National Rail network, and so Realtime Trains does
not allow you to view information for the vast majority of the 270 stations
served by the London Underground.

Although the London Underground has an information system geared towards the
general public, there is currently no publicly-available software to track
individual trains through the system, nor is there a way to view the timetable
(both planned and predicted) from each station more than a few trains in
advance. Instead, the Transport for London website concentrates on ``live
departure board'' style data, showing for each station when the next trains are
due.

Such a tracking system would not only be useful for rail enthusiasts, for
example trying to follow a delayed steam service around the network or to
follow a particular train of interest, but would also be very useful for making
a variety of inferences about the network.

The main source of open data comes from a system called TrackerNet. TrackerNet
is a pull-based system, rather than being push-based, so a request must be made
by the client every time new data is desired. Data is requested over HTTP and
returned in an XML format. To prevent flooding, this data is cached for around
thirty seconds (according to the specification --- in reality, caching up to a
minute is common).

There are four main endpoints available when using this data --- LineStatus,
which returns overview information about the operation of each line (whether
there is disruption or closures), StationStatus which gives similar information
about stations (closures, etc.), PredictionSummary which gives basic train
prediction information on a per-line basis, and PredictionDetailed which gives
more detailed train prediction information. This is requested on a per-station
and per-line basis\cite{TrackerNetSpec}. Therefore, one request per station per
line is required (349 in total) per half minute, should all stations be
desired. When this report refers to TrackerNet, unless otherwise stated the
PredictionDetailed endpoint is meant.

This data contains an attribute called TrackCode, which is the most accurate
source of location data --- it represents the current signalling block in which
the train is located (or more accurately, the current track circuit or axle
counter section which it occupies --- these being two different methods of train
detection). However, there is no publicly-available data to represent the
location of track codes, neither geographical nor topological. The caching of
data and the fact that some track codes are quite short (less than 30 seconds
apart) means that it is less than trivial to construct such a map.

There is another source of data --- the public timetable as used by the TfL
journey planner. It is stored in a format known as TransXChange, a format
developed as a national standard for storing UK bus
timetables\cite{TransXChangeSpec}. Unfortunately, however, this data does not
receive any supplements for engineering works or alterations to the usual
timetable. It also doesn't contain non-passenger trains, like empty workings or
engineering trains, which would be important to enthusiasts. Finally, and most
importantly, as far as I have been able to ascertain there is no way to relate
this data with the live data.

There are also PDF working timetables available\cite{TfLWTT}, which do contain
all trains and do contain IDs which can be cross-referenced with the live data
--- a set number, representing one particular train's workings throughout the
day (though the physical train used to form this set will be different on
different days, the set number is displayed using a dot-matrix LED or
mechanical display in a prominent position on the front of the train); and a
trip number, which increments each time the train terminates and forms a new
working (and is not displayed).

\subsection{Goals and scope}

\begin{enumerate}
  \item To produce a software system to track trains throughout the London
    Underground System and store the results of past days
  \item To infer the working timetable for this data by averaging past runs
  \item To produce a web-based user interface for viewing this data per-station
    or per-train
  \item To expand the system to predict the future arrival and departure times
    of trains, including when forming new services
  \item To produce a topological map of track codes, and to use this map to
    further improve the predictions
\end{enumerate}

The main project comprises goals 1--3. Goals 4 and 5 are stretch goals.
Evaluation of the main project can be performed by testing a sample of the
produced timetable manually against the PDF and seeing how well they match.
Evaluation of goal 4 involves seeing how well the predictions match subsequent
reality. Evaluation of goal 5 is more difficult --- if a portion of a track
code map cannot be obtained, checking the map to ensure that at least the track
layout matches reality (maps of the London Underground track layout not
including signalling or track code data dating from November 2008, which for
most purposes is new enough, are commercially available from
TRACKmaps\cite{TRACKmaps5}) would be a reasonable evaluation step.

\section{Background research}

There have been a few analyses published on the state of open data on various
forms of public transport. This section is intended to summarize this state for
railways in particular, including light rail and rapid transit systems like the
London Underground.

The recent Parliamentary Office of Science and Technology
briefing\cite{POSTnote472} states that the Department for Transport currently
publishes 255 datasets, of which 244 are issued under the Open Government
licence, which is a Creative Commons-compatible open data licence. However,
there are a further 481 datasets that are not currently published --- the vast
majority. It also states that data is frequently not retained or archived in
the sector. There is also the issue of the data being usable --- data that is
hard to use, or that is of questionable quality, will not encourage as much
usage as good quality and easy to use data. Fixing these issues may be
expensive and time-consuming, so there is a trade-off involved in opening up
data. It is worth noting that one user of Transport for London's London
Underground open data, the author of Twitter bot @whensmytube, noted that it is
much more difficult to use and less useful than the equivalent data for buses
on his blog\cite{whensmytube}.

However, overprocessing can also be a problem --- it has, for example, been
argued that raw data is often more useful than data pre-processed, perhaps
(intentionally or otherwise) with specific use cases in mind that sometimes
limit use cases that may not have initially been considered\cite{Robinson2009}.
The wide variety of use cases of open data should be kept in mind --- use cases
for railway-related data can be split into three broad
categories\cite{Kuhn2011}: advanced search, useful to allow users to search for
specific attributes that may not be possible using the operator's own website;
mashups, which combine railway data with data from other sources (perhaps other
public transport data to produce a more integrated public transport search
system); and visualisations, for example a heatmap of areas with most delay, or
a map showing where trains are physically located --- the latter of which has
been built for the London Underground using TrackerNet
data\cite{TrainTimesTube}.

Another survey, of accessibility data in particular, notes the difficulties of
relating data from multiple sources, and suggests publishing data as linked
open data to solve this\cite{Ding2014}.

There is also the issue that trains are operated commercially by multiple
companies, which causes data to be inconsistently available, or available but
not under open licences. For example, National Rail Enquiries (a part of the
Association of Train Operating Companies (ATOC), an association of all
passenger operators on the national network) until recently charged to access
their data. They now provide free access to individuals and small
organisations, but still charge a considerable fee for sites which have more
than five million usages in a four week period, meaning it is not true open
data, and is still not usable for sites such as Realtime Trains. This data is
the only source of some information on the National Rail network such as delay
causes and certain methods for train cancellation\cite{CairnsSeminar2013} (of
which there are three, only one of which is available through Network Rail's
open data channels).

\section{Potential approaches}

\section{Final approach and justification}

\section{Work to date}

\section{Remaining work}

\section{Support required}

\section{Gantt chart}

\bibliographystyle{acm}

\bibliography{references}

\end{document}
